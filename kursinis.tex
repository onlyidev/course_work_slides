% Vilnius University beamer template
% Created in 2024 by Joana Katina

\documentclass[12pt]{beamer}
% You can use \documentclass[11pt,aspectratio=169]{beamer} 
% to adjust the aspect ratio to 16:9.

\renewcommand{\figurename}{}
\renewcommand{\tablename}{}
\renewcommand\thefigure{\arabic{figure} pav.}
\renewcommand\thetable{\arabic{table} lentelė.}
\newcommand{\enquote}[1]{„#1“}
\usepackage{pifont}
\newcommand{\cmark}{\ding{51}}%
\newcommand{\xmark}{\ding{55}}%

\usepackage{graphicx}
\usepackage{listings}
\usepackage[
    backend=biber,
    style=numeric,
    sorting=ynt
]{biblatex}
\usepackage{algorithm,algorithmic}
\usepackage{caption}
\usepackage{subfig}

\usepackage{biblatex}
\usepackage{hyperref}
\usepackage{siunitx}
\usepackage{booktabs}

\sisetup{
    output-decimal-marker = {,}
}

\newenvironment{criteriaTable}{
    \newcommand{\rowLast}[1]{##1}
    \newcommand{\row}[1]{##1 \\}
    \newcommand{\tbl}[1]{\gdef\Table{##1}}

    \def\Table{}
}{
    \begin{table}[h]
        \centering
        \begin{tabular}{l|c|c|c|S}
            \row{
            \textbf{Karkasas} & 
            \textbf{K-1}      & 
            \textbf{K-2}      & 
            \textbf{K-3}      & 
                \textbf{K-4}*
            } \midrule
            \Table{}
        \end{tabular}
        % \caption{Karkasų vertinimas pagal kriterijus. Karkasai surikiuoti pagal įvertinimą mažėjančia tvarka.}\label{tab:criteria}
    \end{table}
}

\addbibresource{zotero.bib}
\captionsetup{justification=centering}

\usetheme{Madrid}

\title[]{Varžymosi principais pagrįstų atakų karkasų, tinkamų kenkėjiško kodo obfuskacijai, analizė}
\subtitle[]{Kursinis darbas}
\author[Liudas Kasperavičius]{Liudas Kasperavičius}
\date{}

%% Darbo vadovas
\addtobeamertemplate{author}{}{Darbo vadovas: prof. dr. Olga Kurasova\par}

\setbeamertemplate{navigation symbols}{}
\titlegraphic{\includegraphics[width=2cm]{resources/MIF.png}}

\usepackage{VUMIF}
\usepackage[T1]{fontenc}

\begin{document}

\begin{frame}
    \titlepage
\end{frame}

\section{Kenkėjiškų programų aptikimas}
\begin{frame}
    \frametitle{Kenkėjiškų programų aptikimas}
    \begin{center}
        \huge\textbf{450000} naujų kenkėjiškų programų per dieną \large{\cite{MalwareStatisticsTrendsa}}
    \end{center}\pause

    \vspace{40pt}
    Aptikimui naudojami:
    \begin{itemize}
        \item Programų pėdsakai\pause
        \item \textbf{Dirbtinis intelektas}
    \end{itemize}
\end{frame}

\section{Tyrimo apimtis}
\begin{frame}
    \frametitle{Tyrimo apimtis}
    \framesubtitle{Scope}

    Varžymosi principais grįstos atakos skirstomos į 3 atvejus:
    \begin{enumerate}
        \item \textbf{Baltos dėžės} atvejis.
        \item \textbf{Juodos dėžės su pasitikėjimo įverčiu} atvejis.
        \item \underline{\textbf{Juodos dėžės} atvejis:} kenkėjiško kodo kūrėjas gali tik testuoti modelį. Atsakymo forma yra tik klasifikacija.
    \end{enumerate} \pause
    \vspace{10pt}

    \textbf{Tikslas} -- nustatyti labiausiai tinkantį karkasą varžymosi principais pagrįstoms atakoms \enquote{juodos dėžės} atvejais. \pause

    \textbf{Uždaviniai}:
    \begin{enumerate}
        \item Apžvelgti kenkėjiško kodo obfuskacijos metodus.
        \item Nustatyti kriterijus varžymosi principais grįstų atakų karkasams ir juos
              įvertinti.
        \item Atlikti eksperimentinį tyrimą su vienu iš įvertintų karkasų ir patikrinti
              vertinimo rezultatus.
    \end{enumerate}

\end{frame}

\section{Varžymosi principais grįstos atakos}
\begin{frame}
    \frametitle{Varžymosi principais grįstos atakos}
    \framesubtitle{\textit{Adversarial Attacks}}
    \begin{figure}
        \begin{small}
            \begin{center}
                \includegraphics[width=0.60\textwidth]{resources/decision_boundaries.png}
            \end{center}
            \caption{Sprendimų priėmimo ribos \cite{VisualisingDecisionBoundaries}}
            \label{fig:decision_boundaries}
        \end{small}
    \end{figure}\pause

    \vspace{-10pt}

    \begin{columns}[b]
        \column{0.5\textwidth}
        \begin{figure}
            \begin{small}
                \begin{center}
                    \includegraphics[width=\textwidth]{resources/adversarial_example.png}
                \end{center}
                \caption{Sėkminga ataka \cite{AdversarialImagesAttacks}}
                \label{fig:adversarial_example}
            \end{small}
        \end{figure} \pause

        \column{0.5\textwidth}
        \begin{figure}
            \begin{small}
                \begin{center}
                    \includegraphics[width=\textwidth]{resources/malware_adversarial.png}
                \end{center}
                \caption{Atakos kenkėjiškų programų kontekste}
                \label{fig:malware_adversarial}
            \end{small}
        \end{figure}
    \end{columns}

\end{frame}

\section{Karkasai}
\begin{frame}
    \frametitle{Karkasai}
    \framesubtitle{Frameworks}
    Karkaso sudedamosios dalys:\pause
    \begin{itemize}
        \item ML modelio tipas ir architektūra, \pause
        \item surogatinio modelio (jei naudojamas) tipas ir architektūra, \pause
        \item ML modelio naudojami požymiai, \pause
        \item ML modelio naudojamos perturbacijos.
    \end{itemize}\pause
    \vspace{20pt}
    \begin{columns}[t]
        \column{0.4\textwidth}
        \centering{Generatyvinių priešiškų tinklų karkasai}
        \vspace{10pt}
        \begin{itemize}
            \item \textit{MalGAN}
            \item \textit{N-gram MalGAN}
            \item \textit{MalFox}
        \end{itemize}\pause
        \column{0.3\textwidth}
        \centering{Skatinamojo mokymosi karkasai}
        \vspace{10pt}
        \begin{itemize}
            \item \textit{DQEAF}
            \item \textit{MalInfo}
        \end{itemize}\pause
        \column{0.3\textwidth}
        \centering{Genetinių algoritmų karkasai}
        \vspace{10pt}
        \begin{itemize}
            \item \textit{AIMED}
            \item \textit{GAMMA}
        \end{itemize}
    \end{columns}
\end{frame}

\section{Karkasų vertinimas}
\begin{frame}
    \frametitle{Karkasų vertinimas}
    \begin{enumerate}[K-1.]
        \item Karkasas pritaikytas \enquote{juodos dėžės} atvejams
              (\textbf{kokybinis}).\pause
        \item Karkasas pritaikytas komerciniams detektoriams (\textbf{kokybinis}).\pause
        \item Karkasas užtikrina realistišką atakos efektyvumo įvertinimą -- naudoja
              surogatinį modelį (\textbf{kokybinis}).\pause
        \item Karkasas užtikrina atakos efektyvumą (\textbf{kiekybinis}).
    \end{enumerate}\pause

    \begin{criteriaTable}
        % \begin{tabular}{cols}
        \tbl{
            \row{ \textit{MalFox}        & \cmark{} & \cmark{} & \cmark{} & 56,00 \;\%}
            \row{ \textit{MalInfo}       & \cmark{} & \cmark{} & \xmark{} & 59,40 \;\%}
            \row{ \textit{AIMED}         & \cmark{} & \cmark{} & \xmark{} & 47,98 \;\%}
            \row{ \textit{N}-gram MalGAN & \cmark{} & \xmark{} & \cmark{} & 88,58 \;\%}
            \row{ \textit{MalGAN}        & \cmark{} & \xmark{} & \cmark{} & 81,00 \;\%}
            \row{ \textit{GAMMA}         & \cmark{} & \xmark{} & \xmark{} & 53,00 \;\%}
            \rowLast{ \textit{DQEAF}     & \cmark{} & \xmark{} & \xmark{} & 46,56 \;\%}
        }
        % \end{tabular}
    \end{criteriaTable}
    \footnotesize{* Atakų efektyvumo ar ekvivalentus įvertis, pateikiamas karkaso straipsnyje}

\end{frame}

\begin{frame}
    \frametitle{\textit{MalFox} ir \textit{MalGAN} sąlygų suvienodinimas}
    \begin{figure}
        \begin{small}
            \begin{center}
                \includegraphics[width=0.5\textwidth]{resources/mu_distribution.png}
            \end{center}
            \caption{\textit{MalGAN} atakų efektyvumas prieš komercinius detektorius}
        \end{small}
    \end{figure} \pause

    Vidurkis ($\bar{\mu}$) = 26,85\;\% \pause \\ $\implies$ Kriterijų vertinimas
    adekvatus \pause \\ $\implies$ \textit{MalFox} karkasas geriausiai tinka
    varžymosi principais pagrįstoms atakoms \enquote{juodos dėžės} atvejais

\end{frame}

\section{Rezultatai ir išvados}
\begin{frame}
    \fontsize{9pt}{10pt}\selectfont
    \frametitle{Rezultatai ir išvados}
    \textbf{Rezultatai}
    \begin{enumerate}
        \item Išanalizuoti 7 trims kategorijoms (\uppercase{gan}, \uppercase{rl}, \uppercase{ga})
              priklausantys varžymosi principais pagrįstų atakų karkasai. 
        \item Jiems apibrėžti vertinimo kriterijai, išskiriantys geriausiai realioms sąlygoms
              pritaikytus karkasus.
        \item Atliktas tyrimas, kurio metu K-2 kriterijaus
              neįgyvendinantis karkasas \textit{MalGAN} naudojamas \uppercase{ae}
              generavimui prieš komercinius detektorius ir renkami
              K-4 kriterijų atitinkantys duomenys (t.~y. nustatomas
              karkaso K-4 kriterijus tokiomis sąlygomis, kai tenkinamas K-2).
    \end{enumerate}

    \textbf{Išvados}
    \begin{enumerate}
        \item Tyrimo metu atlikti 2 eksperimentai parodo, jog net ir padidinus
              \textit{MalGAN} \uppercase{ml} modelio galimybes (suteikus daugiau kompiuterijos
              resursų), atakų efektyvumas prieš komercinius detektorius nepadidėja
              ($\bar{\mu}_{E \text{-} 2} < \bar{\mu}_{E \text{-} 1}$), tačiau yra stabilesnis
              ($\sigma_{E \text{-} 2} < \sigma_{E \text{-} 1}$).
        \item Abiem atvejais tyrimo metu gautas K-4 įvertis
              ($\bar{\mu}$) yra žymiai mažesnis už kriterijų vertinimo metu naudojamą $81,00
                  \; \%$ \textit{MalGAN} karkaso įvertinimą (kai
              K-2 netenkinamas). Tai paaiškina vertinimo lentelėje matomas K-4
              kriterijaus anomalijas ir parodo, jog kriterijai ir karkasų
              vertinimas yra adekvatūs.
        \item Laikant, jog parinkti kriterijai ir jų vertinimas yra patikimi, geriausiai iš
              analizuotų karkasų \enquote{juodos dėžės} atvejams pritaikytas
              karkasas yra \textit{MalFox}, tenkinantis visus 3 kokybinius
              kriterijus ir siekiantis $56,00 \; \%$ kiekybinį įvertinimą (atakų efektyvumą).
    \end{enumerate}

\end{frame}

\section{Literatūros šaltiniai}
\begin{frame}[t,allowframebreaks]
    \frametitle{Literatūros šaltiniai}
    \printbibliography
\end{frame}

\end{document}
