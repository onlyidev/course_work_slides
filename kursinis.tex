% Vilnius University beamer template
% Created in 2024 by Joana Katina

\documentclass[12pt]{beamer}
% You can use \documentclass[11pt,aspectratio=169]{beamer} 
% to adjust the aspect ratio to 16:9.

\renewcommand{\figurename}{}
\renewcommand{\tablename}{}
\renewcommand\thefigure{\arabic{figure} pav.}
\renewcommand\thetable{\arabic{table} lentelė.}

\usepackage{graphicx}
\usepackage{listings}
\usepackage[
    backend=biber,
    style=numeric,
    sorting=ynt
]{biblatex}
\usepackage{algorithm,algorithmic}
\usepackage{caption}
\usepackage{subfig}

\usepackage{biblatex}
\usepackage{hyperref}
\usepackage{tikz}

\usetikzlibrary{arrows.meta, positioning}
\addbibresource{zotero.bib}
\captionsetup{justification=centering}

\usetheme{Madrid}

\title[]{Varžymosi principais pagrįstų atakų karkasų, tinkamų kenkėjiško kodo obfuskacijai, analizė}
\subtitle[]{Kursinis darbas}
\author[Liudas Kasperavičius]{Liudas Kasperavičius}
\date{}

%% Darbo vadovas
\addtobeamertemplate{author}{}{Darbo vadovas: prof. dr. Olga Kurasova\par}

\setbeamertemplate{navigation symbols}{}
\titlegraphic{\includegraphics[width=2cm]{resources/MIF.png}}

\usepackage{VUMIF}
\usepackage[T1]{fontenc}

\begin{document}

\begin{frame}
    \titlepage
\end{frame}

\section{Kenkėjiškų programų aptikimas}
\begin{frame}
    \frametitle{Kenkėjiškų programų aptikimas}
    \begin{center}
        \huge\textbf{450000} naujų kenkėjiškų programų per dieną \large{\cite{MalwareStatisticsTrendsa}}
    \end{center}\pause

    \vspace{40pt}
    Aptikimui naudojami:
    \begin{itemize}
        \item Programų pėdsakai\pause
        \item \textbf{Dirbtinis intelektas}
    \end{itemize}
\end{frame}

\section{Varžymosi principais grįstos atakos}
\begin{frame}
    \frametitle{Varžymosi principais grįstos atakos}
    \framesubtitle{\textit{Adversarial Attacks}}
    \begin{figure}
        \begin{small}
            \begin{center}
                \includegraphics[width=0.60\textwidth]{resources/decision_boundaries.png}
            \end{center}
            \caption{Sprendimų priėmimo ribos \cite{VisualisingDecisionBoundaries}}
            \label{fig:decision_boundaries}
        \end{small}
    \end{figure}\pause

    \vspace{-10pt}

    \begin{columns}[b]
        \column{0.5\textwidth}
        \begin{figure}
            \begin{small}
                \begin{center}
                    \includegraphics[width=\textwidth]{resources/adversarial_example.png}
                \end{center}
                \caption{Sėkminga ataka \cite{AdversarialImagesAttacks}}
                \label{fig:adversarial_example}
            \end{small}
        \end{figure} \pause

        \column{0.5\textwidth}
        \begin{figure}
            \begin{small}
                \begin{center}
                    \includegraphics[width=\textwidth]{resources/malware_adversarial.png}
                \end{center}
                \caption{Atakos kenkėjiškų programų kontekste}
                \label{fig:malware_adversarial}
            \end{small}
        \end{figure}
    \end{columns}
\end{frame}

\begin{frame}
    \frametitle{Karkasai}
\end{frame}

\begin{frame}
    \frametitle{Karkasų vertinimas}
\end{frame}

\section{Literatūros šaltiniai}
\begin{frame}[t,allowframebreaks]
    \frametitle{Literatūros šaltiniai}
    \printbibliography
\end{frame}

\end{document}
